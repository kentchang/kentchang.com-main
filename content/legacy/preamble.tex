%%%%%%%%%%%%%%%%%%%%%%%%%%%%%%%%%%%%%

%\makeatletter
%\renewcommand{\maketitle}{\bgroup\setlength{\parindent}{0pt}
%\begin{flushleft}
%  \@title
%
%  \@author
%\end{flushleft}\egroup
%}
%\makeatother

%%%%%%%%%%%%%%%%%%%%%%%%%%%%%%%%%%%%%
% Font and typography 
%%%%%%%%%%%%%%%%%%%%%%%%%%%%%%%%%%%%%

% LetterSpace=10

% Font
\usepackage{microtype}
\usepackage{fontspec}
\setmainfont[Ligatures=TeX,%
	Numbers=OldStyle,%
	BoldFont={* Semibold},%
	UprightFeatures={%
	SmallCapsFeatures={Letters=SmallCaps,LetterSpace=7},%
	},%
	BoldFeatures={%
	SmallCapsFeatures={Letters=SmallCaps,LetterSpace=7},%
	},%
	]{Garamond Premier Pro}



\setsansfont[%
	BoldFont={* Semibold}, %
	UprightFeatures={%
	SmallCapsFont={Acumin Pro Medium},%
	SmallCapsFeatures={Letters=SmallCaps,LetterSpace=10},%
	},%
	BoldFeatures={%
	SmallCapsFont={Acumin Pro Bold},%	
	SmallCapsFeatures={Letters=SmallCaps,LetterSpace=10},%
	},%
	Scale=0.82]{Acumin Pro}

\setmonofont[Scale=0.85]{Source Code Pro}

\newcommand\tracked[1]{%
  {\addfontfeature{LetterSpace=10}#1}}
  
\newcommand{\allcaps}[1]{%
  {\addfontfeatures{LetterSpace=10}\MakeUppercase{#1}}%
}  


% Spacing

\usepackage{setspace}
\onehalfspacing % 130% spacing between lines: http://texblog.org/2011/09/30/quick-note-on-line-spacing/
\setlength{\parskip}{0.25\baselineskip}
\setlength\parindent{0pt} % indentation


% Titles
%
%\usepackage[sc,small,raggedright,compact]{titlesec}
%\titleformat{\section}%
%	{\sffamily\bfseries}%
%	{}%
%	{}%
%	{\addfontfeatures{LetterSpace=10}%
%		\MakeUppercase}
%\titleformat*{\subsection}{\scshape\lowercase}
%\newcommand{\periodafter}[1]{#1.}
%\titleformat{\subsubsection}[runin]%
%    {\itshape}{\thesubsubsection}{0pt}{\periodafter}

% URL

\usepackage[dvipsnames]{xcolor}
\usepackage[
  pdftitle={Cultural and Literary Text Mining: General Information},
  pdfauthor={Kent K. Chang},
  bookmarks, bookmarksopen,
  colorlinks=true,urlcolor=blue,citecolor=BlueViolet,
  xetex]{hyperref}
%\urlstyle{same}

% Misc

\usepackage[english]{babel}
\usepackage{xunicode}
\usepackage{xltxtra} 
\usepackage{hanging}
\usepackage{fancybox}

%%%%%%%%%%%%%%%%%%%%%%%%%%%%%%%%%%%%%%%%%
% Layout
%%%%%%%%%%%%%%%%%%%%%%%%%%%%%%%%%%%%%%%%%

\usepackage{xkeyval}
\usepackage{enumitem}
\setitemize[1]{label=\raisebox{0.25ex}{\tiny$\bullet$}}

\setitemize[2]{leftmargin=0.4cm,label=\raisebox{0.25ex}{\tiny$\circ$}}

\usepackage{multicol}

% Margin, space, etc.

\usepackage[lmargin=1.2in,rmargin=1.2in]{geometry} % wider margins
\geometry{hcentering=true,xetex}

% header, footer

\usepackage{fancyhdr}
\pagestyle{fancy}

\renewcommand{\footrulewidth}{0 pt} % I don't like fancyhdr's rules.
\renewcommand{\headrulewidth}{0 pt}
\fancyhead{}
\fancyhead[LO]{\small {\scshape \runningHeaderL}}
\fancyhead[CO]{\small }
\fancyhead[RO]{\small \runningHeaderR}
\fancyfoot{}
\cfoot{\small \thepage}
% VCDateUSA macro supplied in tweaked vc-git.awk
\rfoot{\small Last revised: \today}

%%%%%%%%%%%%%%%%%%%%%%%%%%%%%%%%%%%%%%%%%
% accomodating biblatex-chicago
%%%%%%%%%%%%%%%%%%%%%%%%%%%%%%%%%%%%%%%%%

% Some tweaks necessary to ensure annotations print at the end of 
% bibliography entries.
\usepackage{csquotes}
\usepackage[authordate,strict,backend=biber,autolang=other,bibencoding=inputenc,booklongxref=false,compresspages,doi=false,isbn=false,url=false,cmsdate=both]{biblatex-chicago}
\DeclareFieldFormat{annotation}{#1\isdot}
%\usepackage{usebib}

% Swaps biblatex-chicago's default for \autocite from \footcite to \cite.
% This way, when pandoc converts [@key] citations to \autocite, you'll
% get inline (short) citations.
\DeclareAutoCiteCommand{footnote}{\cite}{\cites}

% Bibliography will print \small
\renewcommand{\bibfont}{\small}

\addbibresource{../../docs/course-2.bib}


\setcounter{secnumdepth}{-2}	% Suppress section numbers even with unstarred
				% (sub)section commands.

%%%%%%%%%%%%%%%%%%%%%%%%%%%%%%%%%%%%%%%%%
\usepackage{cancel} 
\usepackage{amssymb}
\usepackage{amsmath}
\usepackage{xifthen}
\usepackage{tcolorbox}
