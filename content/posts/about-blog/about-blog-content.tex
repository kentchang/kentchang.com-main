For some very curious reason I have had to change my blogging platform every now and then since I was 12 (seriously---I was tweaking pLog on my 13th birthday), so here I am. 

It's not like I hated all the previous platforms I used. The previous version of my website is basically a few static pages plus a Pelican blog, which is good enough for my purpose in terms of functionality. The issue, it seems to me, is always appearance. I created that Pelican blog a year ago, and I learned Sass \textit{after} that. Sass is my savior, although it has turned the Pelican theme I edited from a hot mess to a terrifying hot mess. So I decided to give that up.

And for some reason, I find Pelican tiring, and out of my control. This doesn't sound sensible; I guess I simply wanted to try another static site generator. I then found Gatsby.js, which is based on React.js. React, Gatsby, GraphQL---those sound much nicer than Python, Pelican, Anaconda, etc. to me, who studied American literature in college and regularly use Photoshop and Illustrator.

Of course I could go analytical---Pelican is in many ways a good system, and it's written in my favorite language. But I just find the writing process tedious. I had to type a few commands which generate a Markdown file, do the writing, publish, upload, etc. I thought I would get used to it (that's how static sites work), and 90\% of the things I do tend to be more complex. But the reality is I am reluctant to even start writing.

Pick whatever reason you like. But as it turns out, the irony is I have to do more or less the same thing with Gatsby here too. Originally, I decided to use Contentful with Gatsby, but later gave up, not for technical reasons, but because I simply prefer to have a local copy of Markdown files, which I also used with Pelican. So, the difference is not really the platform. It's me. I'm now much more comfortable with geeky stuff, and I've found smarter ways to do things. My workflow now roughly looks like this:

\begin{itemize}
	\item run \texttt{gen.py} which generates \texttt{.tex} and \texttt{.md} header (front matter) files.
	\item write in the \texttt{.tex} file
	\item run \texttt{md.py } to use pandoc to convert \texttt{.tex} to \texttt{.md} and then join it with its header \texttt{.md} file
	\item edit the combined \texttt{.md} file as I see fit
	\item run \texttt{push.py} to push changes to Github
	\item let Netlify deploy it
	\item compile the \texttt{.tex} file if there's a need.
\end{itemize}

\noindent There's nothing too complicated here, and since I don't know when I will have time to write a tutorial of sorts, please just go ahead to my Github if you're interested in how this works out.

I think the current website looks better than the previous one. It has a more consistent color scheme and layout, both absent before. I will have to implement more components for the blog, but I guess for the time being it's good enough. The background image on the landing page was inspired by \textit{The Marvelous Mrs. Maisel} and I used the same font they used for the show for my name. 